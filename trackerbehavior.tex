\chapter{Node Beaconing Behavior}

Due to the fact that APRS is a source-routed protocol, most of the decisions as
to how often a station should send traffic and how far that traffic should
travel are allowed to be made by that originating station. 
The existing specification for the protocol hardly touches on this issue, 
while a great deal of time and energy is spent on the development forums 
bemoaning specific examples of misbehaving members of the network.
The two major parameters under the source node's control that are considered
here are the frequency of beaconing and the routing path used for each beacon.

Being a source-routed protocol does fundamentally introduces a
moral hazard in the network. For each node in the network to enjoy the
most benefit from the network, they would want to beacon as often as possible 
with the longest path allowed by the network.
It is only by mutual trust, respect, and education that this 
logic isn't universally followed and the network is not grossly oversubscribed
in the self-interest of every individual network node. 
Unfortunately, the most popular documentation on APRS fails to stress the 
importance of correctly configuring nodes in the best interest for the network
at large, and rarely gives any concrete guidance on what values should typically
be used when configuring various types of network nodes.


\section{Beaconing Algorithms}

For each APRS node with information available for the network,
a local decision needs to be made as to when that information will best 
serve the local network and should be transmitted. This is rarely a trivial 
decision, and one that could warrent much more creative and application or data
specific solutions than the ones presented here, which should be considered 
the typical minimum of most popular APRS trackers. The only datum considered 
in this paper is that of a mobile node's position, but these algorithms would
likely extend to most other user applications of APRS.

\subsection{Fixed Interval Beaconing}

The simplest beaconing algorithm consists of waiting a single fixed interval
between beacons, and only requires a single parameter that is the beacon interval.
When a tracker is turned on, it aquires a GPS lock and immediately sends out a 
beacon and starts a timer. Once that timer exceeds the beacon interval value, a 
new position is aquired from the GPS receiver and the new location is beaconed.
While simple, this algorithm does suffer from a number of inadequacies:
\begin{itemize}
\item The decision to beacon is made solely based on how long it has been
since the previous beacon, without considering any other information available
to the tracker. Examples of additional information would include whether the 
tracker has moved, how fast the tracker is moving, or any packets received from
the APRS network since the last beacon
\item A single fixed interval limits the amount of entropy introduced into the
network with regards to inter-packet arrival at other nodes. Since APRS is a 
CSMA shared channel network, it's tempting to use Poisson distribution models
for network capacity, which is likely invalid when the only source of entropy 
per tracker is the time when it was last turned on or gained GPS lock.
\end{itemize}

There are of course several possible extensions to the fixed interval beaconing
algorithm which each fix various deficiencies at the expense of simplicity.

\subsection{Time Slot Interval Beaconing}

Arguably a more restrictive form of fixed interval beaconing, time slotting is 
based on the idea of preventing channel collisions by allocating each tracker
a fixed time slot in each interval for when they are allowed to transmit.
An infeasible solution for the national APRS network due to its scale and lack
of coordination, time slotting is often applied where unusually high levels of
coordination do exist, such as special events and insular networks.

Time slotting introduces a new parameter called the slot time, and depends on every
tracker using it having syncronized real time clocks, which is reasonable since most
GPS receivers provide real time to within typically 200ms of UTC 
as part of their position reports
\footnote{The 200ms uncertainty is a typical value due to the delivery of time over
an asyncronous 4800 baud serial port. Internally, GPS receivers must maintain 
their real time clocks to several orders of magnitude higher precision than this
to be even remotely useful, but this precision isn't needed for APRS time slotting}.

The slot time determines how many seconds after the beginning of each interval
a tracker should beacon. The beginning of each interval is defined by the top of
the hour, and intervals run successively for the remainder of the hour.
For example, a tracker configured to time slot with an interval of 550 seconds and
a time slot of 12 would beacon at the following times:
\begin{itemize}
\item 00:00:12
\item 00:09:22
\item 00:18:32
\item 00:27:42
\item 00:36:52
\item 00:46:02
\item 00:55:12
\item 01:00:12
\end{itemize}

This deterministic beaconing algorithm allows carefully designed networks to 
over-subscribe the network well beyond the levels expected from 
the national stocastic APRS network. 
On an insular network seperate from the national
network, it would be possible to set an aggresively short beacon interval and 
carefully space trackers such that no two beacons are within two seconds of each
other. This would make it possible to accomplish service levels such as
every-minute position updates from up to 30 tracked vehicles, 
at the expense that there is no allowance for any additional traffic 
on the RF channel, and the network would depend on it being manually 
assured that every tracker is configured to use its proper time slot.

\subsection{Nice Interval Beaconing}

Nice beaconing is a behavioral extension to fixed interval beaconing 
where trackers consider whether a ``echo"
of a position beacon is heard back from any near-by digital repeaters.
Since digipeaters tend to have much higher power transmitters and better quality
antennas than mobile trackers, once a packet is successfully received by any 
digipeater, that packet is much more likely to be received by a much larger
fraction of the target audience. Most implementations introduce a new parameter
called nice \cite[p.~38]{ot3manual}, 
which is the number of subsequent beacons to skip when a digipeater echo is heard.

\subsection{Dithered Interval Beaconing}

\subsection{Smart Beaconing}

\section{Path Recommendations}

\begin{itemize}
\item Fixed site: WIDE2-1 or literal Digis
\item Mobile: WIDE1-1,WIDE2-1
\item Airbore: No path
\item Weather Stations: WIDE2-2?
\item Proportional pathing
\end{itemize}

