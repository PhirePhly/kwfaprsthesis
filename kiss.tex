\chapter{KISS}

KISS (``Keep It Simple, Stupid") is a protocol originally presented by 
Mike Chepponis, K3MC and Phil Karn, KA9Q at the 6th ARRL Computer Networking
Conference in Redondo Beach, CA \cite{KISSspec}.
KISS was designed as an extension to the Serial Line 
Internet Protocol (SLIP) allowing for in-band signalling from 
the host to the modem to enable setting modem 
configuration parameters such as the preamble length and CSMA parameters.

During the early 1980's when amateur Terminal Node Controllers were developed,
the expectation was that the TNC would be handling 
the entire packet protocol stack up to the final presentation to the user, which
could be done using a dumb terminal such as a VT100 or line printer 
and a keyboard. 
As personal computers became affordable in the late 1980's, the expectation that
the entire application stack would run on the embedded TNC became severely limiting
and KISS emerged as the solution to expose the modems inside TNCs via 
an eight bit clean interface and bypass the TNC's internal network stack.

This meant that the existing TNCs with their radio interfaces could
be upgraded with new ROMs that supported KISS and any new network behavior
or protocol could be instead implemented on the host PC.
This was particularly valuable since personal PCs were much
more productive development environments than the 256kb EPROMs and 8 bit
Z80 microprocessors of the popular Tucson Amateur Packet Radio TNC 2 product
and it's several clones \cite{TNC2manual}.

\section{KISS Parameters}
\label{sec:kissparm}

TODO: Document DWait, PPers, slottime, TXDelay

\section{Advantages of KISS}

While originally designed as a stop-gap solution, KISS has continued to be 
the most popular modem to host communication protocol used in amateur packet radio.
This is arguably due to its simplicity, flexibility,
and the foresight on the designer's part
to including means for device-specific extensions to the protocol.

The simplicity of KISS makes it easy to implement on severely resource-constrained
microcontrollers. The designers did an admirable job of keeping the scope of the KISS
specification small such that it lends itself well as the binary transport
layer between two local nodes while being flexible enough to be useful for
more sophisticated systems.

Every KISS command is prefixed by a four bit port address and four bit 
command code. The four bit address means that a single serial port can 
support multiplexing up to 16 radio modems, though implementations with more
than one or two ports are unusual. The command codes indicate the contents of the 
following KISS frame, be it a data frame to be transmitted, a configuration
parameter, or a hardware specific command.

The hardware specific ``SetHardware" command code is particularly valuable
since it enables future developers of hardware using the KISS protocol
that happen to need the ability to send additional configuration parameters
to the modem beyond the four that are documented are able to do so in a
standards-compliant way.

\section{Disadvantages of KISS}

While not fatal, there are a small number of deficiencies with the KISS
protocol that keep it from being an ideal link protocol between hosts and modems.

KISS makes no provisions for checksums or error detection of frames transfered
between host and modem. This is not a problem for the typical system
using a short RS-232 link where bit errors are unexpected.
Unfortunately, other applications of KISS have required checksums, which
have been implemented via
several less-well-documented extensions to KISS, such as SMACK\cite{smack}, 
which see limited application or support.

KISS makes no provisions for other channel access methods other than
Carrier Sense Multiple Access, which limits its usefulness on amateur networks
using other channel access methods such as Time Division Multiple Access
or Demand Assigned Multiple Access, which are more popular in European 
packet networks but are also used in some insular APRS networks 
(see section~\ref{subsec:timeslot})

There are no supported command codes from modem to host other than the data frame
command, which prevents hosts from being able to interogate the device 
about out-of-band information such as the modem's model and feature set,
the current values of channel access parameters, or any measure of the 
channel occupancy other than successfully received complete packets.


