
\chapter{Reference CRC-16-CCITT Implementation}
\label{sec:crcref}
\texttt{
\lstinputlisting{src/crccalc.c}
}
\chapter{SLottime Justification}
\label{sec:vhftxrx}

The SLottime parameter of modems is dependent upon the total time
it takes for one station to begin transmitting and for receiving stations
to subsequently identify the channel as occupied.
This sequence can be broken into the following stages:
\begin{itemize}
	\item Transmitter key-up from standby (RX-TX turnaround time)
	\item RF propagation from transmitter to receiver (negligible)
	\item Receiver opening squelch and delivering AFSK signal to modem
		(While not usually measured, a representative measurement is
		the TX-RX turnaround time \cite{pattersoninterview})
	\item Modem Data Carrier Detect (DCD) of the received signal
\end{itemize}

\begin{center}
\begin{tabular}{ | c || c | c || c |}
	\hline
	Radio Model & RX-TX Time & TX-RX Time & QST Issue \\ \hline
	Yaesu FT-2600M & 55ms & 60ms & Dec 1999 \\ \hline
	Icom IC-910H & 32ms & 70ms & May 2001 \\ \hline
	Kenwood TM-271A & 72ms & 88ms & Mar 2004 \\ \hline
	Yaesu FT-7800R & 190ms & 98ms & Apr 2004 \\ \hline
	Yaesu FT-8800R & 120ms & 190ms & May 2005 \\ \hline
	Icom ID-800H & 55ms & 173ms & Nov 2005 \\ \hline
	Kenwood TM-V708A & 56ms & 86ms & Apr 2006 \\ \hline
	Yaesu FT-1802M & 77ms & 130ms & Jun 2006 \\ \hline
	Kenwood TM-V71A & 75ms & 102ms & Nov 2007 \\ \hline
	Icom IC-2820H & 43ms & 110ms & Nov 2007 \\ \hline
	Kenwood TM-D710A & 75ms & 106ms & Feb 2008 \\ \hline
	Yaesu FT-1900R & 74ms & 150ms & May 2010 \\ \hline
	Yaesu FTM-350R & 134ms & 70ms & Jan 2011 \\ \hline

\end{tabular}
%\caption{RX-TX and TX-RX turnaround times for VHF mobile radios as published in QST}
\end{center}

Since most Amateur Bell 202 packet activity is done using amateur VHF mobile radios,
a survey was performed of ARRL QST magazine hardware reviews published on VHF-capable
mobile radios since 1999.
These reviews included RX-TX and TX-RX turnaround times
which indicate that a significantly longer SLottime is needed than the 
traditionally suggested 100ms. 
This is due to \emph{none} of the evaluated radios
being fast enough to key up, a second radio to open squelch, 
and then allow any time for modem DCD before the end of a slot.
It is the author's opinion that a new default of 300ms SLottime 
for modems be considered.

