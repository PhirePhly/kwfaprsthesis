\part{OSI Layer 1 --- Physical}

The physical layer of APRS defines the different signalling methods used
to transport datalink layer packets from one node to another across one of
the several physical channels typically used in the APRS network.


\chapter{Bell 202 on RF}

The most common layer 1 used for APRS on RF is a variant of the
Bell 202 audio frequency shift keyed (AFSK) modem transmitted via 
FM VHF radios. In North America, the primary frequency of operation is
144.390MHz, but differs by country based on local band limitations.

Bell 202 was originally  based on switching between 1200Hz and 2200Hz tones to
represent a binary one or zero respectively. Due to Amateur radio operators
using Bell 202 as a physical layer below AX.25, which ultimately derived 
from HDLC, the original 1200Hz mark and 2200Hz space symbols are not used;
the layer 2 data stream is encoded using 
non-return to zero, inverted (NRZI),
which requires zeros in the original bit stream to be encoded as a change
between 1200Hz and 2200Hz and ones to be encoded as transmitting the same
symbol during the next symbol period as during the prior.

\section{Stages of a Bell 202 Transmission}

Channel idle state; string of zeros for clock sync

At least one flag octet 0x7E

AX.25 UI frame with bit stuffing

Frame Check Sum; 16 bit CCITT CRC

At least one flag octet 0x7E

Release of channel or an additional AX.25 frame and flag. 

\section{Baseband Performance of Bell 202 Modems}

TODO No FEC is severe limitation

TODO Emphasis / deemphasis problem is unsolvable

TODO Define Basic 3002 phone channel as good benchmark

\section{Carrier Sense Multiple Access}
\label{sec:bell202csma}

Since amateur Bell 202 is a half duplex modulation traditionally using FM voice 
transceivers, one of the challenges to packet radio is avoiding multiple stations
transmitting on the same channel at once. Other derivatives of ALOHAnet, such as 
the original half duplex 10Mbps Ethernet, have the advantage that transmitters can
at least sense when a collision has taken place and use that information to abort
the transmission of the rest of the frame.

Throughout the history of APRS, there has been several debates as to the implications
of CSMA algorithms and primarily the need for one at all versus an entirely 
stocastic channel access method. The argument follows that the majority of 
channel contention is between multiple clients trying to reach a single digipeater
across what's called a split horizon, where each of the clients can hear and be heard
by the digipeater, but are entirely unaware of each other. In these types of instances,
even an optimal CSMA algorithm will never correctly cause one of the stations to hold 
off until the end of the other's transmission due to the needed information being
entirely unavailable.

Besides the degenerate case of ignoring the current channel status completely when 
deciding to transmit a pending frame, there are two major algorithms used for CSMA;
DWait and P-persistent.

DWait is a deterministic algorithm where each station is assigned a fixed
``quiet time" after the end of another tranmission before they will begin a locally
pending tranmission. This lends itself well to very carefully designed networks
where the relative priority of each station is known and a corresponding DWait time
is set for each station where a shorter DWait will always gain the channel over a longer
one. The requirement for the network to have an overarching design doesn't lend itself
well to the national APRS network, but could be applied effectively for localized 
portions of the APRS network and for ``insular" networks built for special events or
private groups of amateurs.

P-persistent is a statistical algorithm with two variables: the slot time, and the
probability that a station should transmit at the beginning of a slot. The slot time
should presumably be set to as short of a time interval during which a station can
reliabily identify another station as transmitting before beginning its own transmission,
and the P value for how likely a station is to transmit should be set based on the 
number of other stations with pending traffic attempting to gain the channel.

TODO Table of typical mobile radio tx-rx and rx-tx changeover times.
TODO The typical 100ms slot time seems to be much too short.

\chapter{KISS}

During the early 1980's when amateur Terminal Node Controllers were developed,
the expectation was that the TNC would be handling 
the entire packet protocol stack up to the final presentation to the user, which
could be done using a dumb terminal such as a VT100 or line printer 
and a keyboard. 
As personal computers became affordable in the late 1980's, the expectation that
the entire application stack would run on the embedded TNC became severely limiting
and KISS (``Keep It Simple, Stupid") emerged as the solution to 
expose the modem inside TNCs via 
an eight bit clean interface and bypass the TNC's internal network stack.

KISS was originally presented by 
Mike Chepponis, K3MC and Phil Karn, KA9Q at the 6th ARRL Computer Networking
Conference in Redondo Beach, CA \cite{KISSspec}.
KISS was designed as an extension to the Serial Line 
Internet Protocol (SLIP) allowing for in-band signaling from 
the host to the TNC to enable setting modem 
configuration parameters such as the preamble length and CSMA parameters.

This meant that the existing TNCs with their radio interfaces could
be upgraded once with new ROMs that supported KISS and any new network behavior
or protocol could be instead implemented on a seperate host PC.
This was particularly valuable since personal PCs were much
more productive development environments than the 256kb EPROMs and 8 bit
Z80 microprocessors of the popular Tucson Amateur Packet Radio TNC 2 product
and it's several clones \cite{TNC2manual}.

\section{Isolating Modulation from Network}

The advantage of KISS is that it has become the standard packet interchange protocol
between TNC modems and host network controllers,
enabling each side to experiment with new protocols.
This means that as the APRS protocol has evolved, stations that used the
KISS protocol are able to continue using the same ROM-based modems and only
need to upgrade the software running on their host system.
This abstraction holds up even further in that it allows operators to
use different data link protocols than AX.25,
yet there have been few examples of this since the collapse of the IPv4 amateur
radio networks with the wide-spread deployment of the Internet.

As discussed in the prior chapter,
as Bell 202 and HDLC begin to show their age, the field is ripe for a new
modulation to replace them on VHF/UHF packet networks.
Should a researcher wish to deploy a new modulation to use under APRS,
all that needs to be done is build new KISS modems, which will seamlessly
interface with most existing APRS software.
Replacing Amateur Bell 202 modems has always been a popular subject of
discussion on the APRS mailing lists, and is an area ripe for future
quantitative study.

\section{Shortcomings of KISS}

While originally presented as an interim solution until a better protocol was
developed, KISS has enjoyed a lasting popularity among its users.
One concern about KISS that has spawned several derivative protocols is the
lack of a checksum used in each KISS frame.
Should any bit errors happen between the host system and the modem,
they may go undetected and cause corruption in the transported payload
as it continues through the network.
One of the most popular of these derivatives of KISS is SMACK (Stuttgart Modified
Amateurradio-CRC-KISS), which is a backwards compatible extension to KISS
which includes a frame checksum to protect against frame corruption \cite{smack}.

Traditionally, KISS links between the host PC and KISS modems have been deployed
over relatively short RS-232 serial links (three to six feet),
though an increasing number of modems are moving to a pure USB implementation.
What the author has been unable to find is any evidence that when correctly
deployed that these types of short serial links stand any risk of corruption
which could be avoided by using the SMACK extension.
Further study is needed to examine typical APRS KISS deployments to justify
the additional effort required to implement and deploy SMACK above KISS to
protect against any possible corruption risks.

A second shortcoming of KISS which has not evoked anywhere near the
level of discussion that the corruption issue has is KISS' lack of any way
for modems to pass out-of-band (OOB) information back to the host PC.
KISS defines numbered command codes for the host to pass information to the
modem, including channel access settings and hardware specific settings
beyond the type `0' data frame that should be transmitted on the channel.
In the opposite direction, from modem to host, the KISS specification explicitly
limits frame types to only data frames.
This disallows hosts from being able to interogate modems as to their
current configuation settings or any information about the RF channel
beyond packets as they are received.

The author suggests that a revision to the KISS specification be considered
that would allow modems to pass non-zero payloads back to the host PC.
Legacy applications would need to properly discard these non-zero frames
as OOB data that they are not interested in,
while enabling new applications such as interactive configuration programs
and give more meaningful metrics as to current channel utilization.
\begin{itemize}
	\item An interactive configation program would allow a user
		to request a dump of the current configuration, edit it, and
		re-upload it to one or several KISS modems. A canonical example
		of this type of application is the OTWINCFG tool 
		provided by Argent Data for their trackers \cite[\S17.6]{ot3manual}.
	\item Channel utilization information could include how much time
		is consumed on the channel by AFSK data that does not decode as
		valid packets. Some APRS applications attempt to estimate channel
		utilization based on received packets, which will consistently
		under-estimate the actual figure due to frame preambles
		and frames which fail their checksums never being reported.
		The worst case of continuous collisions would result in
		the application misinterpretting no received packets as
		an entirely idle channel, where the opposite is the actual case.
\end{itemize}

\section{Conclusion}

KISS have proven itself tremendously useful as it has decoupled the modem
hardware from the network protocol used above it.
The decreasing price and size of single-board computers such as the
BeagleBone have cemented the popularity of network nodes built using
a full-fledged computer that uses a KISS modem as a radio interface.
It should be appreciated that KISS enables the development of new
modulation schemes without the need to modify existing APRS software
to run over new higher-quality links.
Future work involving KISS should involve studying the quanitative risks
of not using checksums on the serial link between the host and the modem
and the feasibility of extending KISS to allow OOB information to be passed
from modem to host.



\chapter{Bell 103 and PSK63}

TODO Mention the two main HF modes. 

