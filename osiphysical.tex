\part{OSI Layer 1 --- Physical}

The physical layer of APRS defines the different signalling methods used
to transport datalink layer packets from one node to another across one of
the several physical channels typically used in the APRS network.


\chapter{Bell 202 on RF}

The most common layer 1 used for APRS on RF is a variant of the
Bell 202 audio frequency shift keyed (AFSK) modem transmitted via 
FM VHF radios. In North America, the primary frequency of operation is
144.390MHz, but differs by country based on local band limitations.

Bell 202 was originally  based on switching between 1200Hz and 2200Hz tones to
represent a binary one or zero respectively. Due to Amateur radio operators
using Bell 202 as a physical layer below AX.25, which ultimately derived 
from HDLC, the original 1200Hz mark and 2200Hz space symbols are not used;
the layer 2 data stream is encoded using 
non-return to zero, inverted (NRZI),
which requires zeros in the original bit stream to be encoded as a change
between 1200Hz and 2200Hz and ones to be encoded as transmitting the same
symbol during the next symbol period as during the prior.

\section{Stages of a Bell 202 Transmission}

Channel idle state; string of zeros for clock sync

At least one flag octet 0x7E

AX.25 UI frame with bit stuffing

Frame Check Sum; 16 bit CCITT CRC

At least one flag octet 0x7E

Release of channel or an additional AX.25 frame and flag. 

\section{Baseband Performance of Bell 202 Modems}

TODO No FEC is severe limitation

TODO Emphasis / deemphasis problem is unsolvable

TODO Define Basic 3002 phone channel as good benchmark

\section{Carrier Sense Multiple Access}

Since amateur Bell 202 is a half duplex modulation traditionally using FM voice 
transceivers, one of the challenges to packet radio is avoiding multiple stations
transmitting on the same channel at once. Other derivatives of ALOHAnet, such as 
the original half duplex 10Mbps Ethernet, have the advantage that transmitters can
at least sense when a collision has taken place and use that information to abort
the transmission of the rest of the frame.

Throughout the history of APRS, there has been several debates as to the implications
of CSMA algorithms and primarily the need for one at all versus an entirely 
stocastic channel access method. The argument follows that the majority of 
channel contention is between multiple clients trying to reach a single digipeater
across what's called a split horizon, where each of the clients can hear and be heard
by the digipeater, but are entirely unaware of each other. In these types of instances,
even an optimal CSMA algorithm will never correctly cause one of the stations to hold 
off until the end of the other's transmission due to the needed information being
entirely unavailable.

Besides the degenerate case of ignoring the current channel status completely when 
deciding to transmit a pending frame, there are two major algorithms used for CSMA;
DWait and P-persistent.

DWait is a deterministic algorithm where each station is assigned a fixed
``quiet time" after the end of another tranmission before they will begin a locally
pending tranmission. This lends itself well to very carefully designed networks
where the relative priority of each station is known and a corresponding DWait time
is set for each station where a shorter DWait will always gain the channel over a longer
one. The requirement for the network to have an overarching design doesn't lend itself
well to the national APRS network, but could be applied effectively for localized 
portions of the APRS network and for ``insular" networks built for special events or
private groups of amateurs.

P-persistent is a statistical algorithm with two variables: the slot time, and the
probability that a station should transmit at the beginning of a slot. The slot time
should presumably be set to as short of a time interval during which a station can
reliabily identify another station as transmitting before beginning its own transmission,
and the P value for how likely a station is to transmit should be set based on the 
number of other stations with pending traffic attempting to gain the channel.

TODO Table of typical mobile radio tx-rx and rx-tx changeover times.
TODO The typical 100ms slot time seems to be much too short.

\chapter{KISS}

KISS (``Keep It Simple, Stupid") is a protocol originally presented by 
Mike Chepponis, K3MC and Phil Karn, KA9Q at the 6th ARRL Computer Networking
Conference in Redondo Beach, CA in 1987 as an extension to the Serial Line 
Internet Protocol (SLIP) described in RFC 1055 allowing for in-band signalling from 
the host to the Terminal Node Controller (TNC) to enable setting modem 
configuration parameters. 

During the early 1980's, the expectation was that the TNC would be handling 
the entire packet protocol stack up to the final presentation to the user, which
could conceivably be done using a dumb terminal such as a VT100 or line printer 
and a keyboard. 
Once personal computers became affordable in the late 1980's, the expectation that
the entire application stack would run on the embedded TNC became severely limiting
and KISS emerged as the solution to expose the modems inside TNCs via 
an eight bit clean interface.

TODO Describe typical application of KISS in node.

TODO Being 8 bit clean is a huge advantage.

TODO Doesn't allow for TNC to host message passing

\chapter{Bell 103 and PSK63}

TODO Mention the two main HF modes. 

