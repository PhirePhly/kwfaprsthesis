\chapter{Preface}

Like any major research endeavor, this thesis certainly didn't start anywhere near
where it ended.
Most of the credit for the genesis for this thesis needs to be given to 
Sivan Toledo, 4X6IZ, from Tel-Aviv University. In 2012 he wrote an article in the
amateur radio QEX technical journal where he explored improving the soundcard 
digital signal processing modem used for amateur packet radio by passing the 
original signals through a series of band-pass filters \cite{sevanhighperf}. 
His improvements were
commendable, and his article was very well written, but what bothered me was that 
more than three decades after the inception of amateur packet radio,
we are still seeing
measurable improvement in the modems we use for the original modulation techniques.

This thesis started with me wanting to tear apart the current state of the various
Bell 202 modems used in amateur radio, build a quantitative model of the kinds of
interference and distortion that each modem handled well, and hopefully design
a new signal processing algorithm that showed immunity to the most common forms of
interference on real-world channels. Sevan's work in JAVA showed promise, but 
while his library is useful on desktop computers and Android devices, it left 
out in the cold the many different 8, 16, and 32 bit fixed-point microcontrollers
that are often used for embedded modems in amateur radio projects.

As I started to examine the specifications for the various network layers used
in the amateur Automatic Packet Reporting System (namely Bell 202, AX.25, and
APRS), I grew increasingly shocked and confused when I kept finding that the
documentation I was looking for was poorly written or simply didn't exist. 
Protocol specifications would identify variables critical for network performance,
and then never give guidance on what the actual value should be.
Many of the documents
on the expected behavior for network nodes consist solely of console commands
to be run on specific pieces of discontinued hardware instead of actual protocol
behavior.
Most articles discussing aspects of the network disagreed with 
other documentation on specific details,
and was often internally inconsistent as well.

The final turning point was an interview in March, 2014 with Scott Miller, the
designer for Open Trackers, 
which are one of the more popular lines of contemporary modems used in the APRS
network.
I brought a laundry list of inconsistencies from the network specs and he
explained how much effort he had put into reverse engineering the existing 
hardware. It was an eye-opening conversation that drove home how much the 
amateur packet network has grown haphazardly over the past three decades into
a jumble of band-aids applied upon band-aids.

I realized that the most important academic research on the topic of APRS isn't
how to squeeze out another incremental improvement in one of the modem DSP
algorithms, but an over-arching prolegomenon on the entire network stack as it 
actually exists today. The existing documentation clearly falls short, and
much of the institutional knowledge that I've been able to draw on during 
my research is coming from the ``old guard" of the hobby, which leaves us exposed
to the labor intensive requirement of newcomers to the network to 
reverse engineer the existing network before they can participate.

Ideally, this document would be able to stand by itself as a complete 
``implementer's guide to APRS" from the physical layer all the way up to high level
aggregate network behavior, but the scope of reverse engineering that much
behavior, documenting it, and then verifying the documentation quickly becomes
monumental. 
It is my hope that this document does at least identify the most glaring 
short-falls in the current documentation and network design, and gives answers to
the questions that can be answered while staying within the scope of this survey.

For every identified problem which is answered in this paper, there
are twice as many unanswered questions which each warrant being considered 
as a thesis of their own.


\chapter{Introduction}

The Automatic Packet Reporting System (APRS) is an amateur radio packet
network designed to provide each participating node a local view of the 
tactical environment based on each node beaconing it's current status
and advertising any other local resources known to exist.

Exactly what types of resources should be advertised on a local APRS
network is left to the discretion of the local network coordinators, but
a typical APRS network would advertise information such as:
\begin{itemize}
\item The location of amateur radio operators and what frequencies they are using for voice communications.
\item The location, frequency, and access information for voice repeaters
\item The location and status of APRS digital repeater nodes
\item The location and access information for other packet networks such as BBSes, Winlink nodes, or open Internet access points.
\item The location and status of useful facilities such as rest stops, resupply points for food/water, etc.
\item Telemetry from sensors such as weather stations or remote site monitoring (i.e. battery voltage at solar sites)
\item Short real-time messages and annoucements directed at other amateur radio operators
\end{itemize} 

Despite these flexible capabilities, 
and much to the chagrin of many of the designers of APRS, 
the vast majority of user traffic on the APRS network consists solely of 
real-time automatic vehicle tracking (AVT).
Fittingly, it follows that one of the hotbeds for APRS network congestion
is the Los Angeles basin, 
due to its bowl-shaped geography and unusually high population density.
When discussing specifics of the APRS network, such as how often to send traffic
or how many hops to route it over, LA invariably comes up as a counter-example
that under-cuts any specific guidance on what to expect from the network.

The author is more interested in being able to make concise statements about 
APRS in general than construct an entirely exhaustive analysis, 
so the reader need only appreciate that LA is the exception to the rules.
Any readers operating in the LA basin have the author's heart-felt condolences,
but need look elsewhere for definitive guidance on operating in such a unique
part of the APRS network.

\section{Goals of This Document}

One of the greatest failings of amateur packet radio is the lack of definitive
collections of documentation on the complete protocol stacks currently used in 
networks such as APRS. This is understandable as the task of documenting a 
system as complex as APRS is momumental and requires digging through decades of 
relevant standards, magazine articles, and personal diatribes made by amateurs
infulential in the field.

The body of this document is divided into three sections loosely based on the 
three lowest layers of the OSI seven layer network model. Many portions of the APRS
protocol stack
pre-date the OSI model, and maintaining clean separation of layers even in 
contemporary protocols is challenging, so the lines drawn in this document should
be viewed with a level of forgiveness by the reader. More importantly, these three
layers are drawn to aid future work on experimenting with physical modulation 
and data link protocols underneath the APRS packet application.

While some of these differentiations already exist in the standing literature,
this document isn't holding itself bound to the existing implicit conventions.
One example is the likely controversial separation made between Bell 202 and 
AX.25 where I (the author?) have gone as far as to
remove the Frame Checksum (FCS) from AX.25 packets all together and place them
instead in the Bell 202 frame, motivated by the fact that AX.25 packets are moved
across other physical channels such as KISS without the required FCS.

\section{History of APRS}

APRS was created as an evolution of the AX.25 packet networks
built throughout the amateur community during the 1980s and 1990s and 
the Connectionless Emergency Traffic System (CETS) built by 
Bob Bruninga during the early 1980s to map Navy position reports.

Near-ubiquitious access to the Internet caused the decline in local BBS 
systems and AX.25 TCP/IP networks during the 1990s, but APRS has 
continued to enjoy a growing userbase due to it filling a unique 
application of amateur packet radio to local short-lived communication.

APRS supports basic communication between stations via node to node 
text messages and comment field status updates, but should not be 
considered a communications network to an end, but a way to be made aware
of the other assets in the local area made available to support amateur 
radio operations.

Due to the fact that APRS is built upon the relatively slow 
1200bps AX.25 VHF packet
network and the channel sharing concepts developed for the ALOHAnet at
the University of Hawaii, the amount of 
traffic and the number of stations that it is possible to successfuly 
support on a single regional network is severly limited. 
A tyical APRS network is considered successful if a single node
can use it to discover the 60 closest other assets on the network in a
10-30 minute time frame. Trying to advertise information beyond this
``ALOHA circle" consisting of the 60 closest stations exceeds the 
operational objective of the APRS network and usually proves to only be 
detrimental to the network and other users as network throughput is 
consumed by advertisements for 
resources beyond the radius of interest for the local operator.

