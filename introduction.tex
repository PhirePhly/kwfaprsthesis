\chapter{Introduction}

The Automatic Packet Reporting System (APRS) is an amateur radio packet
network designed to provide each participating node a local view of the 
tactical environment based on each node beaconing it's current status
and advertising any other local resources known to exist.

Exactly what types of resources should be advertised on a local APRS
network is left to the discretion of the local network coordinators, but
a typical APRS network would advertise information such as:
\begin{itemize}
\item The location of amateur radio operators and what frequencies they are using for voice communications.
\item The location, frequency, and access information for voice repeaters
\item The location and status of APRS digital repeater nodes
\item The location and access information for other packet networks such as BBSes, Winlink nodes, or open Internet access points.
\item The location and status of useful facilities such as rest stops, resupply points for food/water, etc.
\item Telemetry from sensors such as weather stations or remote site monitoring (i.e. battery voltage at solar sites)
\end{itemize} 

\section{History of APRS}

APRS was created as an evolution of the AX.25 packet networks
built throughout the amateur community during the 1980s and 1990s and 
the Connectionless Emergency Traffic System (CETS) built by 
Bob Bruninga during the early 1980s to map Navy position reports.

Near-ubiquitious access to the Internet caused the decline in local BBS 
systems and AX.25 TCP/IP networks during the 1990s, but APRS has 
continued to enjoy a growing userbase due to it filling a unique 
application of amateur packet radio to local short-lived communication.

APRS supports basic communication between stations via node to node 
text messages and comment field status updates, but should not be 
considered a communications network to an end, but a way to be made aware
of the other assets in the local area made available to support amateur 
radio operations.

Due to the fact that APRS is built upon the relatively slow 
1200bps AX.25 VHF packet
network and the channel sharing concepts developed for the ALOHAnet at
the University of Hawaii, the amount of 
traffic and the number of stations that it is possible to successfuly 
support on a single regional network is severly limited. 
A tyical APRS network is considered successful if a single node
can use it to discover the 60 closest other assets on the network in a
10-30 minute time frame. Trying to advertise information beyond this
``ALOHA circle" consisting of the 60 closest stations exceeds the 
operational objective of the APRS network and usually proves to only be 
detrimental to the network and other users as network throughput is 
consumed by advertisements for 
resources beyond the radius of interest for the local operator.


