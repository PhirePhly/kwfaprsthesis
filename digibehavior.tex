\chapter{Digital Repeater Routing Behavior}

An integral part of APRS is the huge array of digital repeaters deployed throughout the network.
The job of the digipeaters is to receive, process, and selectively retransmit packets from other 
stations in the network to improve network coverage and each station's effective range.
Originally, since AX.25 was used almost exclusively for static networks, it was valid that
the networks relied on the concept of manually constructing the exact string of digipeater hops
between two stations. When one of these digipeaters subsequently went down, a new route would 
need to be found between the two stations.

One of the large inovations of APRS was that these literal routing paths be replaced with 
well-known universal aliases for digipeaters such that a single node could move from area to
area and the local infrastructure would still support the same requested routing behavior.

\section{RELAY,WIDE}

The original conception of APRS included several aliases, but the most important were RELAY
and WIDE. Digipeaters were divided into two categories depending on if they were high-level
(on the top of large towers, mountain tops, etc.) or low-level (primarily sub-40' home installs).

Low-level ``fill-in" digipeaters are needed to assist low-power moving trackers in reaching the
primary high-level digipeater network where it would be possible to be heard by other stations.
Without these fill-in repeaters, lower power beacons would be lost in the noise and never reach the 
network at large. Therefore, trackers that need this help would begin their routing path with RELAY.

The high-level digipeaters that consist the major portion of the network coverage in terms of area
would respond to the alias of WIDE, in addition to the alias of RELAY such that if a tracker got
lucky and happened to be decoded by one of the high level digis it wasn't punished for first requesting 
help from a lower-level digi.

This RELAY for low-level and both RELAY and WIDE alias for high-level digipeater concept worked well
because the existing packet hardware at the time natively supported configuring these aliases, but
became increasingly problematic as the APRS network grew and became higher density. Each digipeater
could not verify that they had already retransmitted a beacon, so single packets would ``ping-pong" 
between digipeaters repeatedly until the entire path was finally consumed.

To solve this, a new digipeater behavior was implemented where each digi would keep a log of
all the packets it had retransmitted in the last 30 seconds based on the source callsign and 
packet contents (TODO verify the exact fields needed for dedup behavior). Therefore, when a digipeater 
hears an echo of an already digipeated packet, it will correctly ignore it and not generate more
redundant traffic on the APRS channel.

\section{WIDEn-N}

At the same time as the development of the deduplication concept, a new alias of WIDEn-N was 
developed with the goal of shortening AX.25 headers. A typical packet requesting three hops through
WIDE digipeaters would construct a routing path of ``WIDE,WIDE,WIDE," which would cause the 
AX.25 header to become 21 bytes longer than the minimum 16. The idea of WIDEn-N was to replace this
three hop path with a single alias of ``WIDE3-3," where the first number indicates the originally
requested number of hops and the second number the number of remaining hops. Therefore, as a packet
with an original path of "WIDE3-3" moved through digipeaters, each digipeater would rewrite the path 
to be ``WIDE3-2," ``WIDE3-1," and finally ``WIDE3-0*" at which point the path would be completely 
consumed and should no longer be digipeated.

While this was a sound design at the time, a failure in the implementation of this alias in one of
the more popular firmware revisions of Kamtronics TNCs meant that the deduplication algorithm used to
prevent packet echos from bouncing between digipeaters was not applied between the WIDEn-N and the 
original RELAY alias. Excessive cost to purchase new firmware ROMs from Kamtronics meant that much
of the network had no choice but to find a technical workaround to this firmware bug.

The lasting solution to this issue was proposed by Stephen Smith where the alias of RELAY would be
replaced by an alias of WIDE1-1, which allowed the Kamtronics' TNCs to use their deduplication
algorithms while preserving a literal alias to be used by the lower level digipeaters, many of which
did not and never would support the WIDEn-N alias concept. Low-level digis would respond to WIDE1-1, 
while higher level digis would respond to all permutations of WIDEn-N.

The only shortfall with this proposal to repurpose WIDE1-1 as a replacement for RELAY was that it
left no proper way to request a single hop from high-level digipeaters while excluding the low-level
digis not needed by stationary or high power trackers. The clever solution was to recommend that single
high-level hops be requested by the path of ``WIDE2-1," which literally means that the original tracker
requested two hops and that only one of them is left, which violates the original spirit of the leading
number in the alias but functionally yields the desired behavior.

\section{Digipeater Levels}

TODO This is where I flat out contradict the existing standard?
