\chapter{Digital Repeater Routing Behavior}

When the AX.25 networks were originally built in the 1980s,
one of the fundamental design assumptions was that every node was 
physically static in the network.
Digipeaters were installed on top of high buildings or mountain tops, 
and client nodes were modems connected to bulky video terminals or desktop PCs.
When two stations wanted to exchange packets,
the operators had to manually construct an explicit list of digipeaters to use
to deliver packets to the other station.
Should a station move to a new area, 
the operator would need to discover new near-by digipeaters and manually construct
routing paths using them.

One of the design goals of APRS has been to support mobile nodes,
so this requirement to pre-facto be aware of the local infrastructure is unacceptable.
The solution was to categorize digipeaters into a small number of ``aliases" such that
a digipeater would respond to both its specific callsign and to any of its aliases.
A mobile station expecting to move throughout the APRS network could then construct it's
routing paths purely out of digipeater aliases and use the network infrastructure despite
not knowing each digipeater's callsign or location.

As APRS has grown from an experimental network to one that covers much of the country,
it has adopted and discarded multiple sets of digipeater aliases and sets of expectations
as to each node's behavior regarding those aliases, without making it clear that the
previous behavior has been depricated.
The rest of this chapter will walk through the history of the basic set of routing aliases,
followed by a detailed examination of each digipeater's expected behavior. 

\section{RELAY,WIDE}

The original conception of APRS included several aliases, but the most important were RELAY
and WIDE. Digipeaters were divided into two categories depending on if they were high-level
(on the top of large towers, mountain tops, etc.) or low-level (primarily sub-40' home installs).

Low-level ``fill-in" digipeaters are needed to assist low-power moving trackers in reaching the
primary high-level digipeater network where it would be possible to be heard by other stations.
Without these fill-in repeaters, lower power beacons would be lost in the noise and never reach the 
network at large. Therefore, trackers that need this help would begin their routing path with RELAY.
The low-level digipeaters should only respond to the RELAY alias.

The high-level digipeaters that consist the major portion of the network coverage in terms of area
would respond to the alias of WIDE, in addition to the alias of RELAY, such that if a tracker got
lucky and happened to be decoded by one of the high level digis it wasn't punished for first requesting 
help from a lower-level digi.

This RELAY for low-level and both RELAY and WIDE alias for high-level digipeater concept worked well
because the existing packet hardware at the time natively supported configuring these aliases, but
became increasingly problematic as the APRS network grew and became higher density. Each digipeater
could not verify that they had already retransmitted a beacon, so single packets would ``ping-pong" 
between digipeaters repeatedly until the entire path was finally consumed.

To solve this, a new digipeater behavior called deduplication (dedup) was implemented.
Each digi is expected to store a checksum of every recently heard or transmitted packet
based on the source callsign and packet contents for a limited amount of time. 
Should a digipeater receive a packet that is already in it's ``recently heard" database,
it should not digipeat it again, since that would likely cause additional network traffic
with little benefit.
It's important to note that the concept of a ``unique" packet is based \emph{solely} on the
source address and information field; APRS doesn't guarantee the destination address to be
an invariant, and the routing path will change on each hop.
The original design did not specify how long this dedup window should be,
but general consensus has settled on removing packets from the deduplication database after 30 seconds.

As an example, consider a small APRS network with three digipeaters: LOWDIG, HIGHA, and HIGHB.
LOWDIG is a low level digipeater that only responds to the RELAY alias, 
where HIGHA and HIGHB are both high-level digis and thus respond to both RELAY and WIDE.
Consider the following sequences of routing paths,
given various initial paths originating from a tracker and the assumption that packets are only
heard in the order listed.
\begin{itemize}
	\item Tracker transmits a packet with the path: WIDE
	\item HIGHA digipeats it as: WIDE*
	\item HIGHB drops the packet since it has no more ``unconsumed" hops.
\end{itemize}
To reach further out in the network, the user would append additional WIDE statements to their path:

\begin{itemize}
	\item Tracker: WIDE,WIDE
	\item HIGHA: WIDE*,WIDE
	\item HIGHB: WIDE*,WIDE*
\end{itemize}

If the tracker doesn't happen to reach any of the high-level digis, 
but does reach a low-level one, a path of WIDE,WIDE would do them no good:

\begin{itemize}
	\item Tracker: WIDE,WIDE
	\item LOWDIG drops the packet since the first path alias is not RELAY
\end{itemize}

Thus, a path of RELAY,WIDE would better serve the user,
since they could use two high-level digis should they happen to initially reach one:

\begin{itemize}
	\item Tracker: RELAY,WIDE
	\item HIGHA: RELAY*,WIDE
	\item HIGHB: RELAY*,WIDE*
\end{itemize}

While at the same time taking advantage of low-level digipeaters when needed:
\begin{itemize}
	\item Tracker: RELAY,WIDE
	\item LOWDIG: RELAY*,WIDE
	\item HIGHA: RELAY*,WIDE*
	\item HIGHB drops the packet due to no remaining hops
\end{itemize}

Even though neither high-level digipeater directly heard the tracker,
the low-level digipeater relayed the packet to HIGHA so at least one high-level
digipeater was able to repeat the packet for the rest of the network.

\section{WIDEn-N}

There was an inherent flaw in the RELAY,WIDE alias set
in that there was no way to summarize the requested path.
Every alias in the routing path lengthens the AX.25 frame by 7 octets,
so long paths such as ``RELAY,WIDE,WIDE,WIDE" were a burden on the network
by adding 28 octets to every packet.
WIDEn-N was developed as a method to condense multiple WIDE aliases into a single WIDEn alias.
Therefore, ``WIDE,WIDE,WIDE" could be rewritten as ``WIDE3-3," and each digipeater would
decrement the SSID instead of marking each WIDE alias as consumed until the entire WIDEn-N had
been consumed.

\begin{itemize}
	\item Tracker: WIDE3-3
	\item HIGHA decrements the non-zero SSID and transmits the packet as: WIDE3-2
	\item HIGHB interprets WIDE3-2 as having two hops remaining of the original three, 
		and consumes one: WIDE3-1
	\item HIGHC uses the final hop remaining and marks the entire WIDE3 as consumed: WIDE3*
\end{itemize}

This change only effects the WIDE alias and not RELAY,
since trackers MUST NOT use more than a single RELAY alias only at the beginning of their packet.
This is because it is assumed that high-level digipeaters have full transmit coverage over
the area of low-level digipeaters below them, so using a path such as WIDE,RELAY would
cause all of the low-level digipeaters below a high-level one to needlessly digipeat the packet.

\section{Deprecation of RELAY}

As the APRS network grew and the density of digipeaters increased in the late 1990s and early 2000s,
the deduplication behavior of each digi became increasingly important to the health of the network.
It became not unusual for digipeaters to be in range of three to five other digipeaters,
so without deduplication every packet could travel along an exponential number of
loops available in the densely connected mesh of digipeaters being built for the APRS network.

At this same time, APRS was becoming popular enough among amateur radio operators that
equipment manufacturers such as Kantronics, MFJ, and Kenwood started adding APRS-specific features
into their TNC products.
One of the most popular TNCs used for digipeater sites was the Kantronics KPC-3+,
which turned out to have an anomaly in it's 
version 9.0 firmware ROM regarding deduplication \cite{kpc3bugbulletin}.

The KPC-3+ correctly deduplicated packets routed via the WIDEn-N system,
but failed to correctly add to or consult the dedup database for single-hop
aliases such as RELAY.
This meant that popular routing paths such as ``RELAY,WIDE2-2" could result in
routing loops on the network:

\begin{itemize}
	\item Tracker: RELAY,WIDE2-2
	\item HIGHA doesn't add the packet to it's dedup database: RELAY*,WIDE2-2 
	\item HIGHB: RELAY*,WIDE2-1
	\item HIGHA hears the echo from HIGHB and thinks it hasn't heard the packet before: RELAY*,WIDE2*
\end{itemize}

Kantronics did release a patched ROM v9.1 in 2007, but to insufficient effect.
Getting access to digipeaters at remote radio sites is burdensome,
and physically replacing the 32 pin ROM chip inside the KPC-3+ 
with a \$40 replacement\footnote{Kantronics did offer free v9.1 ROM exchanges to 
any customers who had purchased their TNCs in the past two years} 
proved to be a sufficient barrier that many APRS digipeaters still
suffer from this defect.\footnote{It has been jokely said that once a digipeater
	goes on the air, none of its settings will be changed until either the digipeater
or its owner dies}

Due to this growing population of digipeaters suffering from the Kantronics or other
misinterpretations of the RELAY,WIDE alias system, 
it was proposed that APRS switch to a purely WIDEn-N routing method. 
Instead of low-level digipeaters responding to RELAY,
they should now only process the alias WIDE1-1.
This means that a path such as ``RELAY,WIDE2-2" should now be rewritten as
``WIDE1-1,WIDE2-2." 
To digipeaters aware of this new interpretation, ``WIDE1-1,WIDE2-2" signifies 
requesting one low-level and two high-level hops.
To older digipeaters like the KPC-3+, it appears to be an odd way to request
three WIDE hops compared to ``WIDE3-3," yet the deduplication is still done correctly.

