\chapter{Preface}

Like any major research endeavor, this thesis certainly didn't start anywhere near
where it ended.
Most of the credit for the genesis for this thesis needs to be given to 
Sivan Toledo, 4X6IZ, from Tel-Aviv University. In 2012 he wrote an article in the
amateur radio QEX technical journal where he explored improving the soundcard 
digital signal processing modem used for amateur packet radio by passing the 
original signals through a series of band-pass filters. His improvements were
commendable, and his article was very well written, but what bothered me was that 
more than three decades after the inception of amateur radio, we're still seeing
measurable improvement in the modems we use for the original modulation techniques.

This thesis started with me wanting to tear apart the current state of the various
Bell 202 modems used in amateur radio, build a quantitative model of the kinds of
interference and distortion that each modem handled well, and hopefully design
a new signal processing algorithm that showed immunity to the most common forms of
interference on real-world channels. Sevan's work in JAVA showed promise, but 
while his library is useful on desktop computers and Android devices, it left 
out in the cold the many different 8, 16, and 32 bit fixed-point microcontrollers
that are often used for embedded modems in amateur radio projects.

As I started to dig into the specifications for the various network layers used
in the amateur Automatic Packet Reporting System (namely Bell 202, AX.25, and
APRS), I grew increasingly shocked and confused when I kept finding that the
documentation I was looking for was skimmed over or simply didn't exist. 
Protocol specifications would identify variables critical for network performance,
and then never touch on what the actual value should be. Many of the documents
on the expected behavior for network nodes consist solely of console commands
to be run on specific pieces of discontinued hardware instead of actual protocol
behavior. Every article touching on parts of the network disagreed with 
other documentation on the subject, and was often internally inconsistent as well.

The final turning point was an interview in March, 2014 with Scott Miller, the
designer for one of the more popular lines of contemporary modems used in the
network.
I brought a laundry list of inconsistencies from the network specs and he
explained how much effort he had put into reverse engineering the existing 
hardware. It was an eye-opening conversation that drove home how much the 
amateur packet network has grown haphazardly over the past three decades into
a jumble of band-aids applied upon band-aids.

I then realized that the most needed academic research on the topic of APRS isn't
how to squeeze out another incremental improvement in one of the modem DSP
algorithms, but an over-arching prolegomenon on the entire network stack as it 
actually exists today. The existing documentation clearly falls short, and 
much of the institutional knowledge that I've been able to draw on during 
my research is coming from the ``old guard" of the hobby, which leaves us exposed
to the labor intensive requirement of newcommers to the network to 
reverse engineer the existing network before they can participate.

Ideally, this document would be able to stand by itself as a complete 
``implementer's guide" to APRS from the physical layer all the way up to high level
aggregate network behavior, but the scope of reverse engineering that much
behavior, documenting it, and then verifying the documentation quickly becomes
monumental. However enjoyable it would be to earn a PhD studying and optimizing
a 1200 baud network built in the 20\textsuperscript{th} century, I'm regrettably 
not the person to answer all of the standing questions on the network. 
It is my hope that this document does at least identify the most glaring 
short-falls in the current documentation and network design, and give answers to
the questions that can be most easily answered.

For every problem I attempt to identify and give an answer to in this paper, there
is easily twice as many unanswered questions which each warrant being considered 
as a thesis of their own.
