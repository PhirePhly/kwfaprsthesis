\chapter{APRS Network Layers}

The typical APRS network uses an OSI network stack consisting of a VHF FM 
voice radio transporting a derivative of Bell 202, AX.25 data link frames,
and APRS network layer messages.

\section{Layer 1 --- Bell 202}

The most common layer 1 modulation used in APRS is wide-band FM VHF radios
tuned to 144.390MHz connected to Bell 202 1200 baud AFSK modems. These 
modems use a NRZI variant of Bell 202 where each zero is represented as a
change in frequency and every one is represented as no change in frequency
between the two audio tones 1200Hz and 2200Hz. Due to the lack of any
guarantee of there being spectral energy at the 1200 baud clocking 
frequency and a need for layer 2 framing, the amateur variant of Bell 202 
includes bit stuffing, where any string of more than five ones in the layer
two bitstream modulated by the Bell 202 modem must be replaced by five 
ones, a zero, and then the subsequent ones from the original bitstream.
This supports clock syncronization between the transmitting and receiving
stations, and also guarantees that a sequence of six ones is only ever seen
in the inter-frame flags, which consist of a series of un-bit-stuffed
0x7E octets when a station is keyed up but not sending frame data.

\section{Layer 2 --- AX.25}

AX.25 is a well documented framing protocol used for amateur radio packet
networks which is a derivative of X.25 and HDLC. The modifications to 
the underlying X.25 protocol allow for each station to be identified by 
their radio callsign and a four bit secondary station identifier to allow
operators to run up to 16 nodes under the same callsign in the same 
namespace.

AX.25 provides fields for the source station's callsign, the destination's
callsign, and up to eight digital repeaters that should relay the framed
layer three packet between these two stations. As each digital repeater,
or digipeater, receives and then retransmits a packet, they set the 
"consumed" flag on their callsign in the path header to indicate that 
they have processed the packet.

APRS has deprecated the use of the destination callsign, since there was
originally no concept of directed messages in APRS but only one-to-many
communications advertising resources. The destination station field has
seen many different uses, including the most popular which is using it to
indicate the type of APRS equipment used at the source station.

APRS has also modified AX.25 by adding a commonly known set of aliases 
to be used in the path header in leu of explicit digipeater callsign-SSID
paths as used in traditional AX.25 packet networks. This enables mobile 
stations to move from network to network and not need to edit their
routing paths since every digipeater is expected to respond to the 
appropriate routing aliases and substitute their callsign-SSID upon 
processing the APRS packet.

APRS depricates the AX.25 status flags in the header, meaning that every
APRS frame should indicate that it is an unconnection information frame
with no layer three protocol, which is unfortunate since they should
have arguably requested a unique layer three identifier from the AX.25
standards committee explicitly for APRS. 

\section{Layer 3 --- APRS}

APRS defines a number of different types of messages to be advertised to
the local ALOHA circle including locations, status messages, telemetry data,
weather data, and third-party objects and items of interest. Each type of
APRS packet is identified by their first character, which maps to a 
specific markup based on what kind of information the packet needs to 
include.

