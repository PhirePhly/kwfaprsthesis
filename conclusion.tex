\chapter{Conclusion}

This paper has considered several of the different aspects of the APRS network,
ranging from the low-level modem and channel access methods up to
how nodes should make beaconing and routing decisions.
During this survey, particular attention has been paid to pointing out
deficiencies in the existing documentation while forming 
an unusually large collection of information on the topic of APRS.
Some of these deficiencies have been followed by improvements
and suggestions, while many of the larger shortfalls have
been merely identified for future work.
Due to many limited resources, this paper cannot stand as the definitive
reexamination of the topic of APRS,
but should be seen as a fastidious call to action to step back and
try to re-examine the APRS network in the modern context.

Some of the major contributions of this paper include:
\begin{itemize}
	\item Explicitly reclassifying HDLC framing as part of the modem
		and not the Layer 2 AX.25 stack, particularly with
		regards to the KISS modem interface protocol.
	\item Novel documentation of the CCITT CRC checksum as shown in
		figure \ref{fig:crcccittcode} and Appendix \ref{sec:crcref}.
	\item Summarizing the AX.25 format as it is used for APRS.
	\item Explicitly deprecating the original meaning of the first number
		in the WIDEn-N routing alias.
	\item Presenting a comprehensive list of beacon interval
		algorithms, including a major reformatting of the
		existing SmartBeaconing documentation as presented
		in figure \ref{fig:kwfsmartbeacon}.
	\item Providing guidance on what the default paths should be for
		various categories of APRS stations.
	\item Applying a simple Poisson traffic model to the APRS network
		as a sanity check that the range goals of APRS are viable.
\end{itemize}

While these will likely prove useful to any readers looking to learn about
the internal mechanics of APRS in the interest of building their own nodes
for the network, it is the author's hope that the lasting value of this paper
eventually proves to be its call to action for others to critically 
reexamine the APRS ecosystem.
APRS has received little of the analytic mind-share that other large
computer networks have received over the past few decades.
As the APRS network has grown, many
of the original design decisions made when it was a very small network
have begun to break down as APRS grows into the tens of thousands of active stations.

A re-enumeration of all the possible avenues for further work
derived from this paper would invariably be incomplete.
The interested reader need not look too deeply to find possible subjects
for further research, as evidenced by the author's painful overuse of
phrases like ``beyond the scope of this work."

While currently used primarily as a vehicle tracking system,
the APRS network offers a tantalizing amount of flexibility to lend itself
useful to countless other applications.
Should enough effort be expended to tame the ambiguities left in the
specification of APRS,
it would be positioned to be a tremendously useful asset to the amateur
radio community.
